%        File: milestone.tex
%     Created: Wed Nov 12 04:00 PM 2014 P
% Last Change: Wed Nov 12 04:00 PM 2014 P
%
%\documentclass[a4paper]{report}
\documentclass{cs229report}
\begin{document}
%------------------------------------------------
% Summary Page
%------------------------------------------------
\heading{Summary}
\bigskip
\renewcommand{\arraystretch}{2.0}
\begin{tabular}[]{|l|p{5in}|}
  \hline
  \textbf{Name} & Huafei Wang, Jennifer Wu \\ \hline
  \textbf{Title} & Classification of Human Posture and Movement\\ \hline
  \textbf{Predicting} & We are classifying between 5 different postures and movements 
  (sitting, sitting down, standing, standing up, walking) \\ \hline
  \textbf{Data} & Data is publicly available on UCI's Machine Learning repository.
  There are $165,633$ rows of data, $12$ columns of which are x,y,z accelerometer
  readings for sensors placed at $4$ different locations on the body.\\ \hline
  \textbf{Features} & The feature space is $12$ dimensional, consisting of x,y,z-axis 
  readings of accelerometers of $4$ different sensors.\\ \hline
  \textbf{Models} & We implemented a GDA model and a K-means clustering model.\\ \hline
  \textbf{Results} & 
  GDA: \newline
  \renewcommand{\arraystretch}{1}
  { \footnotesize
    \begin{tabular}[<+position+>]{|l|c|c|c|c|c|} \hline
      \backslashbox{Actual}{Predicted}
      & Sitting & Sitting down & Standing & Standing up & Walking \\ \hline
      Sitting & 3916 & 1126 & 0 & 22 & 0 \\
      Sitting down & 0 & 1112 & 63 & 7 & 2 \\
      Standing & 0 & 601 & 4136 & 0 & 1 \\
      Standing up & 116 & 558 & 454 & 103 & 11 \\
      Walking & 0 & 1257 & 900 & 48 & 2135 \\ \hline
    \end{tabular}
  }

  K-means: \newline
  { \footnotesize
    \begin{tabular}[<+position+>]{|l|c|c|c|c|c|} \hline
      \backslashbox{Actual}{Predicted}
      & Sitting & Sitting down & Standing & Standing up & Walking \\ \hline
      Sitting & 1000 & 73 & 0 & 54 & 27 \\
      Sitting down & 0 & 0 & 0 & 2 & 95 \\
      Standing & 0 & 340 & 1000 & 26 & 859 \\
      Standing up & 0 & 579 & 0 & 253 & 0 \\
      Walking & 0 & 8 & 0 & 665 & 19 \\ \hline
    \end{tabular}
  }

  \\ \hline
  \textbf{Future} & We plan to reduce the errors on the GDA model, 
  and attempt to implement a softmax regression model.
  Also, in these classes since there is little lateral movement,
  lateral measurements on the sensors may actual introduce more noise.
  We will look into this more closely and possibly eliminate some of the
  accelerometer readings from our feature set.
  \\ \hline
  \textbf{Specific Questions} & Do you have any tips for general things to look for when
  errors are large for GDA?\\ \hline
\end{tabular}
\end{document}


