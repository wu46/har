\documentclass[a4paper, 11pt]{article}
\usepackage{amsmath}
\DeclareMathOperator{\argmax}{arg\,max}
\usepackage{multicol}
\usepackage[margin=1in]{geometry}


\begin{document}
%
% paper title
% can use linebreaks \\ within to get better formatting as desired
% Do not put math or special symbols in the title.
\title{Classification of human posture and movement}

\author{
  Huafei~Wang,
Jennifer~Wu}
% make the title area
\maketitle

\begin{abstract}
  Coming soon\ldots
\end{abstract}
\section{Introduction}
Human activity classification has wide reaching applications,
such as in providing medical assistance to disabled or elderly persons.
This project implements several machine learning algorithms to 
classify human posture and movements. The different activities being classified are: 
\begin{itemize}
  \item Sitting
  \item Sitting down
  \item Standing
  \item Standing up
  \item Walking
\end{itemize}
The difference between ``Sitting'' and ``Sitting down'' is that the former
is the static posture, whereas the latter is the transitional movement
from standing to sitting.
% You must have at least 2 lines in the paragraph with the drop letter
% (should never be an issue)
\subsection{Past work}
Coming soon\dots
\section{Data}
\subsection{Source}
The data is made publicly available on UCI's Machine Learning Repository.
It can be accessed at: 
http://groupware.les.inf.puc-rio.br/har\#dataset.

\subsection{Description}
The dataset contains the following features
\begin{itemize}
  \item Age
  \item Weight
  \item Body Mass Index
  \item Height
  \item x,y,z axis readings from 4 different accelerometers
\end{itemize}

\begin{table}[!h]
  \centering
  \caption{Frequency of each class}
  \begin{tabular}[]{|l|c|} \hline
    \bf {Class} & \bf {Frequency} \\ \hline
    Sitting & 50631 \\ \hline
    Sitting down & 11827 \\ \hline
    Standing & 47370 \\ \hline
    Standing up & 12415 \\ \hline
    Walking & 43390 \\ \hline
  \end{tabular}
\end{table}

% ============================= Features =====================================
\section{Features}
The features used in our models are the $12$ accelerometer readings. Although the original data also contains age, weight, body mass index and height, they are neglected in this preliminary analysis and classification because they are less relevant in determining human movement compared with the $12$ accelerometer readings.

% ================================ GDA =======================================
\section{Gaussian Discriminant Analysis}
\subsection{Binary classification}
The first model implemented was the GDA model. For each class,
$90\%$ of the class's data is used as positive training examples,
while $90\%$ from each of the other classes are concatenated 
to be used as negative training examples.
\subsection{Multi-class classification}
For a testing example, in order to make a prediction into 
$1$ of the $5$ classes, the posterior distribution is calculated for
each class, and the predicted label is chosen depending on the largest
posterior.

That is
\begin{equation*}
  h_\theta(x) = \argmax_y p(x|y)p(y)
\end{equation*} where $y \in \lbrace 1,2,3,4,5 \rbrace$.

\subsection{Results}
The confusion matrix of the results is shown below:

\begin{table}[!h]
  \renewcommand{\arraystretch}{1.5}
  \begin{tabular}[<+position+>]{l|c|c|c|c|c}
  %\backslashbox{Actual}{Predicted}
    & Sitting & Sitting down & Standing & Standing up & Walking \\ \hline
    Sitting & 3916 & 1126 & 0 & 22 & 0 \\
    Sitting down & 0 & 1112 & 63 & 7 & 2 \\
    Standing & 0 & 601 & 4136 & 0 & 1 \\
    Standing up & 116 & 558 & 454 & 103 & 11 \\
    Walking & 0 & 1257 & 900 & 48 & 2135
  \end{tabular}
\end{table}



%\section{Softmax Regression}
%\subsection{}
%\subsection{Results}

% ===================================k-means ==================================
\section{k-means}
\subsection{Training}
The second model implemented is k-means. Although k-means is primarily used in unsupervised learning where there are no labels and we have $5$ labels in this classification problem, because that a particular human movement usually requires a harmonious coordination among different parts of the body, accelerometer readings would exhibit clustering properties. \\
For each class, 5000 data examples are used to train the model, totaling $5000 \times 5 = 25000$ training examples. $30$ iterations are conducted to find the cluster centroids.


\subsection{Model Validation}
Although the accelerometer readings would exhibit clustering properties, due to human movements' complexity, the centroids could be close to each other and thus could cause misclassifications. Therefore it is important to validate the clustering. All training examples are now used as testing examples to check if they have been properly classified.\\
The confusion matrix of the validation is shown below: \\
\begin{table}[!h]
  \renewcommand{\arraystretch}{1.5}
  \begin{tabular}[<+position+>]{l|c|c|c|c|c}
  %\backslashbox{Actual}{Predicted}
    & Sitting & Sitting down & Standing & Standing up & Walking \\ \hline
    Sitting & 5000 & 1883 & 0 & 482 & 286 \\
    Sitting down & 0 & 0 & 0 & 0 & 744 \\
    Standing & 0 & 1742 & 5000 & 1509 & 3959 \\
    Standing up & 0 & 1367 & 0 & 2169 & 7 \\
    Walking & 0 & 8 & 0 & 840 & 4
  \end{tabular}
\end{table} \\
It can be seen that the classifications of Sitting and Standing are precise, classifications if sitting down and walking are poor while the classification of Standing up is moderate. 

\subsection{Results}
$1000$ examples in each calss different from the training examples are used to test the model. The confusion matrix of the testing is shown below: \\
\begin{table}[!h]
  \renewcommand{\arraystretch}{1.5}
  \begin{tabular}[<+position+>]{l|c|c|c|c|c}
  %\backslashbox{Actual}{Predicted}
    & Sitting & Sitting down & Standing & Standing up & Walking \\ \hline
    Sitting & 1000 & 73 & 0 & 54 & 27 \\
    Sitting down & 0 & 0 & 0 & 2 & 95 \\
    Standing & 0 & 340 & 1000 & 26 & 859 \\
    Standing up & 0 & 579 & 0 & 253 & 0 \\
    Walking & 0 & 8 & 0 & 665 & 19
  \end{tabular}
\end{table} \\
It can be seen that the testing results is similar to the validation results, showing very good classifications of sitting and standing but unsatisfactory classifications of sitting down, standing up and walking. This suggests a relatively poor capability for k-means to distinguish among dynamic movements of humans.

\end{document}


